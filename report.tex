\documentclass{article}%
\usepackage[T1]{fontenc}%
\usepackage[utf8]{inputenc}%
\usepackage{lmodern}%
\usepackage{textcomp}%
\usepackage{lastpage}%
%
\title{Credit Card Data Report}%
\date{\today}%
%
\begin{document}%
\normalsize%
\maketitle%
All data labeled "differentially private" in this document satisfies differential privacy for %
$\epsilon$=%
2.0 by sequential composition%
\section{Statistics}%
\label{sec:Statistics}%
\subsection{Basic Averages}%
\label{subsec:BasicAverages}%
The follwing statistics were all generated using the sparse vector technique to determine a clipping parameter for the data, and then generating differentially private sums and counts to find a differentially private average.%
\subsubsection{Average Age}%
\label{ssubsec:AverageAge}%
The average age of all credit card customers is 46.33, the differentially private average age of all customers is 46.22. This gives an error of 0.2236\%.

%
\subsubsection{Average Months on Book}%
\label{ssubsec:AverageMonthsonBook}%
The average months on the book of all credit card customers is 35.93, the differentially private average age of all customers is 36.11. This gives an error of 0.5183\%.

%
\subsubsection{Average Credit Limit}%
\label{ssubsec:AverageCreditLimit}%
The average credit limit of all credit card customers is 8631.95, the differentially private average age of all customers is 8526.09. This gives an error of 1.2264\%.

%
\subsection{Basic Counts}%
\label{subsec:BasicCounts}%
The follwing statistics were all generated using the report noisy max method to determine the highest count in a given parameter.%
\subsubsection{Income Categories}%
\label{ssubsec:IncomeCategories}%
The most common income category of all credit card customers is Less than \$40K. The most common income category as determined by using a differentially private method is Less than \$40K.

%
\subsubsection{Income Categories}%
\label{ssubsec:IncomeCategories}%
The most common education level of all credit card customers is Graduate. The most common education level as determined by using a differentially private method is Graduate.

%
\subsection{Conditional Averages}%
\label{subsec:ConditionalAverages}%
\subsubsection{Average Credit Limit for Customers 33 years old and younger.}%
\label{ssubsec:AverageCreditLimitforCustomers33yearsoldandyounger.}%
The average credit limit of credit card customers who are 33 years old and younger is 7212.7, the differentially private average is 3447.74. This gives an error of 52.199\%.

%
\subsubsection{Average Credit Limit for Customers older than 33 years old.}%
\label{ssubsec:AverageCreditLimitforCustomersolderthan33yearsold.}%
The average credit limit of credit card customers who are over 33 years old is 8719.6, the differentially private average is 8710.11. This gives an error of 0.1088\%.

%
\subsection{Conditional Counts}%
\label{subsec:ConditionalCounts}%
\subsubsection{Most Common Income Category for College Educated Customers}%
\label{ssubsec:MostCommonIncomeCategoryforCollegeEducatedCustomers}%
The most common income level for college educated customers is Less than \$40K, the most common one calculated with a differentially private method is Less than \$40K. 

%
\subsubsection{Most Common Income Category for non College Educated Customers}%
\label{ssubsec:MostCommonIncomeCategoryfornonCollegeEducatedCustomers}%
The most common income level for non college educated customers is Less than \$40K, the most common one calculated with a differentially private method is Less than \$40K. 

%
\end{document}\documentclass{article}%
\usepackage[T1]{fontenc}%
\usepackage[utf8]{inputenc}%
\usepackage{lmodern}%
\usepackage{textcomp}%
\usepackage{lastpage}%
%
\title{Credit Card Data Report}%
\date{\today}%
%
\begin{document}%
\normalsize%
\maketitle%
All data labeled "differentially private" in this document satisfies differential privacy for %
$\epsilon$=%
2.0 by sequential composition%
\section{Statistics}%
\label{sec:Statistics}%
\subsection{Basic Averages}%
\label{subsec:BasicAverages}%
The follwing statistics were all generated using the sparse vector technique to determine a clipping parameter for the data, and then generating differentially private sums and counts to find a differentially private average.%
\subsubsection{Average Age}%
\label{ssubsec:AverageAge}%
The average age of all credit card customers is 46.33, the differentially private average age of all customers is 46.22. This gives an error of 0.2236\%.

%
\subsubsection{Average Months on Book}%
\label{ssubsec:AverageMonthsonBook}%
The average months on the book of all credit card customers is 35.93, the differentially private average age of all customers is 36.11. This gives an error of 0.5183\%.

%
\subsubsection{Average Credit Limit}%
\label{ssubsec:AverageCreditLimit}%
The average credit limit of all credit card customers is 8631.95, the differentially private average age of all customers is 8526.09. This gives an error of 1.2264\%.

%
\subsection{Basic Counts}%
\label{subsec:BasicCounts}%
The follwing statistics were all generated using the report noisy max method to determine the highest count in a given parameter.%
\subsubsection{Income Categories}%
\label{ssubsec:IncomeCategories}%
The most common income category of all credit card customers is Less than \$40K. The most common income category as determined by using a differentially private method is Less than \$40K.

%
\subsubsection{Income Categories}%
\label{ssubsec:IncomeCategories}%
The most common education level of all credit card customers is Graduate. The most common education level as determined by using a differentially private method is Graduate.

%
\subsection{Conditional Averages}%
\label{subsec:ConditionalAverages}%
\subsubsection{Average Credit Limit for Customers 33 years old and younger.}%
\label{ssubsec:AverageCreditLimitforCustomers33yearsoldandyounger.}%
The average credit limit of credit card customers who are 33 years old and younger is 7212.7, the differentially private average is 3447.74. This gives an error of 52.199\%.

%
\subsubsection{Average Credit Limit for Customers older than 33 years old.}%
\label{ssubsec:AverageCreditLimitforCustomersolderthan33yearsold.}%
The average credit limit of credit card customers who are over 33 years old is 8719.6, the differentially private average is 8710.11. This gives an error of 0.1088\%.

%
\subsection{Conditional Counts}%
\label{subsec:ConditionalCounts}%
\subsubsection{Most Common Income Category for College Educated Customers}%
\label{ssubsec:MostCommonIncomeCategoryforCollegeEducatedCustomers}%
The most common income level for college educated customers is Less than \$40K, the most common one calculated with a differentially private method is Less than \$40K. 

%
\subsubsection{Most Common Income Category for non College Educated Customers}%
\label{ssubsec:MostCommonIncomeCategoryfornonCollegeEducatedCustomers}%
The most common income level for non college educated customers is Less than \$40K, the most common one calculated with a differentially private method is Less than \$40K. 

%
\end{document}\documentclass{article}%
\usepackage[T1]{fontenc}%
\usepackage[utf8]{inputenc}%
\usepackage{lmodern}%
\usepackage{textcomp}%
\usepackage{lastpage}%
%
\title{Credit Card Data Report}%
\date{\today}%
%
\begin{document}%
\normalsize%
\maketitle%
All data labeled "differentially private" in this document satisfies differential privacy for %
$\epsilon$=%
2.0 by sequential composition%
\section{Statistics}%
\label{sec:Statistics}%
\subsection{Basic Averages}%
\label{subsec:BasicAverages}%
The follwing statistics were all generated using the sparse vector technique to determine a clipping parameter for the data, and then generating differentially private sums and counts to find a differentially private average.%
\subsubsection{Average Age}%
\label{ssubsec:AverageAge}%
The average age of all credit card customers is 46.33, the differentially private average age of all customers is 46.3. This gives an error of 0.0639\%.

%
\subsubsection{Average Months on Book}%
\label{ssubsec:AverageMonthsonBook}%
The average months on the book of all credit card customers is 35.93, the differentially private average age of all customers is 35.85. This gives an error of 0.2156\%.

%
\subsubsection{Average Credit Limit}%
\label{ssubsec:AverageCreditLimit}%
The average credit limit of all credit card customers is 8631.95, the differentially private average age of all customers is 8638.36. This gives an error of 0.0742\%.

%
\subsection{Basic Counts}%
\label{subsec:BasicCounts}%
The follwing statistics were all generated using the report noisy max method to determine the highest count in a given parameter.%
\subsubsection{Income Categories}%
\label{ssubsec:IncomeCategories}%
The most common income category of all credit card customers is Less than \$40K. The most common income category as determined by using a differentially private method is Less than \$40K.

%
\subsubsection{Income Categories}%
\label{ssubsec:IncomeCategories}%
The most common education level of all credit card customers is Graduate. The most common education level as determined by using a differentially private method is Graduate.

%
\subsection{Conditional Averages}%
\label{subsec:ConditionalAverages}%
\subsubsection{Average Credit Limit for Customers 33 years old and younger.}%
\label{ssubsec:AverageCreditLimitforCustomers33yearsoldandyounger.}%
The average credit limit of credit card customers who are 33 years old and younger is 7212.7, the differentially private average is 3712.21. This gives an error of 48.5324\%.

%
\subsubsection{Average Credit Limit for Customers older than 33 years old.}%
\label{ssubsec:AverageCreditLimitforCustomersolderthan33yearsold.}%
The average credit limit of credit card customers who are over 33 years old is 8719.6, the differentially private average is 8709.78. This gives an error of 0.1126\%.

%
\subsection{Conditional Counts}%
\label{subsec:ConditionalCounts}%
\subsubsection{Most Common Income Category for College Educated Customers}%
\label{ssubsec:MostCommonIncomeCategoryforCollegeEducatedCustomers}%
The most common income level for college educated customers is Less than \$40K, the most common one calculated with a differentially private method is Less than \$40K. 

%
\subsubsection{Most Common Income Category for non College Educated Customers}%
\label{ssubsec:MostCommonIncomeCategoryfornonCollegeEducatedCustomers}%
The most common income level for non college educated customers is Less than \$40K, the most common one calculated with a differentially private method is Less than \$40K. 

%
\end{document}