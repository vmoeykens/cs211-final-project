\documentclass{beamer}

\usepackage[utf8]{inputenc}
\usepackage{listings}

\usetheme{Madrid} 

\makeatother
\setbeamertemplate{footline}
{
  \leavevmode%
  \hbox{%
  \begin{beamercolorbox}[wd=.4\paperwidth,ht=2.25ex,dp=1ex,center]{author in head/foot}%
    \usebeamerfont{author in head/foot}\insertshortauthor
  \end{beamercolorbox}%
  \begin{beamercolorbox}[wd=.6\paperwidth,ht=2.25ex,dp=1ex,center]{title in head/foot}%
    \usebeamerfont{title in head/foot}\insertshorttitle\hspace*{3em}
    \insertframenumber{} / \inserttotalframenumber\hspace*{1ex}
  \end{beamercolorbox}}%
  \vskip0pt%
}

%Information to be included in the title page:
\title{CS211 - Data Privacy: Final Project}
\author{Vincent Moeykens}
\institute{The University of Vermont}
\date{Fall 2020}



\begin{document}

\frame{\titlepage}

% GOALS
\begin{frame}
\frametitle{Project Goals}
This project had a few simple goals:
\begin{enumerate}
\item Identify a data with potential PII
\item Determine which statistics would be valuable from this data
\item Create a sysytem to generate differentially private statistics from this data (using a reasonable privacy budget: $\epsilon$)
\item Automatically generate a PDF report when finished
\end{enumerate}
\end{frame}

% Dataset
\begin{frame}
\frametitle{Dataset}
The dataset I used from this project comes from \textit{Kaggle}. The dataset contains information about credit card customers at some company. \\

While the data itself does not contain simple PII such as names, dates of birth, or SSN; it does contain information such as:
\begin{itemize}
 \item<1-> Customer Age
 \item<2-> Customer Gender
 \item<3-> Education Level
 \item<4-> Marital Status
 \item<5-> Income Level
 \item<6-> \textbf{All things that could be used to re-identify individuals!}
\end{itemize}
\end{frame}

% Statistics
\begin{frame}
\frametitle{Statistics}
I picked a few basic statistics that I thought an analyst might be interested in. They are: 
\begin{itemize}
\item Average Customer Age
\item Average Months on Book
\item Average Credit Limit
\item Count of most common Income Ranges
\item Count of most common Education Level
\item Average Credit Limit of Customers younger than 33y/o vs Average Credit Limit of Customers older than 33y/o
\item Most Common Income Range of Customers with a College Degree vs Customers Without
\end{itemize}
\end{frame}

% Privacy Strategy
\begin{frame}
\frametitle{Privacy Strategy}
To generate the differentially private statistics, I used a few different strategies. 
\newline
\begin{columns}

\column{0.5\textwidth}
For all average statistics, I first use the $above\_threshold$ method to deteremine a upper clipping parameter. I then generate noisy sums and counts using the laplace mechanism (and a portion of the total epsilon alloted for this query). I then divide the noisy sum by the noisy count to get a differentially private average by post processing.

\column{0.5\textwidth}
To determine the category that has the maximum number of occurrences for a parameter, I use the report-noisy-max method. 
\end{columns}
\end{frame}

% Report Generation
\begin{frame}
\frametitle{Report Generation}
Finally, I calculate error percentages for average queries, and verify that the report noisy max results match the expected results. This data then gets automatically generated into a .tex file, and compiled to a pdf (if the user has a LaTeX compiler installed)

\end{frame}

\end{document}