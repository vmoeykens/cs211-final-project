%%%%%%%%%%%%%%%%%%%%%%%%%%%%%%%%%%%%%%%%%
% University/School Laboratory Report
% LaTeX Template
% Version 3.1 (25/3/14)
%
% This template has been downloaded from:
% http://www.LaTeXTemplates.com
%
% Original author:
% Linux and Unix Users Group at Virginia Tech Wiki 
% (https://vtluug.org/wiki/Example_LaTeX_chem_lab_report)
%
% License:
% CC BY-NC-SA 3.0 (http://creativecommons.org/licenses/by-nc-sa/3.0/)
%
%%%%%%%%%%%%%%%%%%%%%%%%%%%%%%%%%%%%%%%%%

%----------------------------------------------------------------------------------------
%	PACKAGES AND DOCUMENT CONFIGURATIONS
%----------------------------------------------------------------------------------------

\documentclass{article}

\usepackage[version=3]{mhchem} % Package for chemical equation typesetting
\usepackage{siunitx} % Provides the \SI{}{} and \si{} command for typesetting SI units
\usepackage{graphicx} % Required for the inclusion of images
\usepackage{natbib} % Required to change bibliography style to APA
\usepackage{amsmath} % Required for some math elements 
\usepackage{hyperref}
\setlength\parindent{0pt} % Removes all indentation from paragraphs

\renewcommand{\labelenumi}{\alph{enumi}.} % Make numbering in the enumerate environment by letter rather than number (e.g. section 6)

%\usepackage{times} % Uncomment to use the Times New Roman font

%----------------------------------------------------------------------------------------
%	DOCUMENT INFORMATION
%----------------------------------------------------------------------------------------

\title{CS 211 Final Project Proposal}

\author{Vincent \textsc{Moeykens}}

\date{December 1, 2020}

\begin{document}

\maketitle

\begin{center}
\begin{tabular}{l r}
Instructor: & Professor Joe Near
\end{tabular}
\end{center}

%----------------------------------------------------------------------------------------
%	SECTION 1
%----------------------------------------------------------------------------------------

\section{Proposal}
For my final project I will be working alone. The goal of this project will be to produce a report on various statistics relating to customers of a credit card company. Credit card data is sensitive information that could lead to re-identification so we will want to use differential privacy to produce these reports. I will look at statistics relating to average Age, most common income level and education level. If time permits, I would also like to look into producing a synthetic representation of one or more columns of the dataset. For the statistics, I will use a few methods, including using the Sparse Vector Technique to determine clipping bounds, then adding noise with the laplace mechanis to compute counts and sums. \\

The dataset I will be using is located here: \href{https://www.kaggle.com/sakshigoyal7/credit-card-customers}{https://www.kaggle.com/sakshigoyal7/credit-card-customers}
\end{document}